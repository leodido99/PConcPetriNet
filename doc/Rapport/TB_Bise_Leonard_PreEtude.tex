\documentclass[11pt,a4paper]{memoir}
\usepackage[utf8]{inputenc}
\usepackage[T1]{fontenc}
\usepackage{times}
\usepackage[french]{babel}
\usepackage{titlesec} % For chapter formatting
\usepackage{tikz}
\usepackage{fancyhdr}
\usepackage{graphicx} % includegraphics
\graphicspath{ {images/} }
\usepackage[
breaklinks=true,colorlinks=true,
linkcolor=black,urlcolor=black,citecolor=black,
bookmarks=true,bookmarksopenlevel=2]{hyperref}  % Links color in pdf
\usepackage{bookmark} % Creates bookmarks in pdf files
\usepackage[
  top=2.5cm,
  bottom=2.5cm,
  left=2.5cm,
  right=2.5cm,
  headsep=3pt,
  footskip=30pt,
  headheight=46pt,
  asymmetric
]{geometry}

% Header
\pagestyle{fancy}
\fancypagestyle{main}{%
  \lhead{\includegraphics[scale=0.5]{HEIG-VD}}
  \rhead{\sffamily Projet Réseau de Petri\\Bise Léonard\\Informatique Embarquée (ISEC)}
  \rfoot{Page \thepage}
  \cfoot{}
}
\fancypagestyle{plain}{%
  \lhead{\includegraphics[scale=0.5]{HEIG-VD}}
  \rhead{\sffamily Projet Réseau de Petri\\Bise Léonard\\Informatique Embarquée (ISEC)}
    \rfoot{Page \thepage}
      \cfoot{}
}

% Modifie la section chapitre
\titleformat{\chapter}
  {\sffamily\LARGE\bfseries}{\thechapter}{1em}{}
% spacing: how to read {12pt plus 4pt minus 2pt}
%           12pt is what we would like the spacing to be
%           plus 4pt means that TeX can stretch it by at most 4pt
%           minus 2pt means that TeX can shrink it by at most 2pt
%       This is one example of the concept of, 'glue', in TeX
\titlespacing*{\chapter}{0pt}{3.5ex plus 1ex minus .2ex}{2.3ex plus .2ex}
% Sets sections to sans serif
\setsecheadstyle{\LARGE\bfseries\sffamily}
\setsubsecheadstyle{\Large\bfseries\sffamily}
\setsubsubsecheadstyle{\large\bfseries\sffamily}
\setparaheadstyle{\large\bfseries\sffamily}
\setsubparaheadstyle{\large\bfseries\sffamily}

\begin{document}

%%%%%%%%%%%%%%%%%%%%% Page de Titre %%%%%%%%%%%%%%%%%%%%%%

\thispagestyle{empty}

\sffamily

% Logo HEIG-VD
\begin{tikzpicture}[remember picture,overlay]
\node[anchor=north west,yshift=-40pt,xshift=40pt]%
        at (current page.north west)
        {\includegraphics[scale=0.8]{HEIG-VD}};
\end{tikzpicture}

\large

~\vspace{\fill}

\begin{center}
{\HUGE 
Gestion d'un carrefour à l'aide d'un réseau de Petri
}
\vspace{1cm}
\hrule
\vspace{1cm}
{\LARGE
Rapport du projet
}
\end{center}

\vspace{3.5cm}

\begin{flushleft}
Auteur: Bise Léonard\\
Professeur: Zysman Eytan\\
Filière: Informatique Embarquée (ISEC)\\
Cours: Programmation Concurrente (PConc)\\
Date: 10 Mai 2018
\end{flushleft}

\vspace{\fill}

\cleardoublepage

%%%%%%%%%%%%%%%%%%%%%% Table des matières %%%%%%%%%%%%%%%%%%%%%%%

\tableofcontents

\clearpage

%%%%%%%%%%%%%%%%%%%%%% Liste des figures %%%%%%%%%%%%%%%%%%%%%%%

\listoffigures  % Write out the List of Figures

\clearpage

%%%%%%%%%%%%%%%%%%%%%% Liste des Tables %%%%%%%%%%%%%%%%%%%%%%%

\listoftables  % Write out the List of Tables

\clearpage

%%%%%%%%%%%%%%%%%%%%%% Introduction %%%%%%%%%%%%%%%%%%%%%%%

\chapter{Introduction}

Dans le cadre du cours Programmation Concurrente (PConc), il nous a été demandé de créer un programme qui simule le comportement d'un carrefour. Le carrefour est composé de deux flux, un Sud-Nord et l'autre Ouest-Est ou les voitures circulent dans un seul sens. A l'intersection des deux flux sont placé deux feux de signalisation, un par flux, ils permettent ou non au voiture de passer.\par
La gestion du système est fait grâce à un réseau de Petri qui réagit aux événement et qui produit des actions.


Test \cite{Reference1}

\section{Test}

This is a test text also

\subsection{Test subsection}

\subsubsection{subsub test}




%%%%%%%%%%%%%%%%%%%%%% Bibliographie %%%%%%%%%%%%%%%%%%%%%%%

\bibliographystyle{unsrt}
\bibliography{bibliographie}  % The references (bibliography) information are stored in the file named "bibliographie.bib"

\end{document}