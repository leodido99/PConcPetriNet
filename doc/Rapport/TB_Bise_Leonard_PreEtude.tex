\documentclass[11pt,a4paper, openany]{memoir}
\usepackage[utf8]{inputenc}
\usepackage[T1]{fontenc}
\usepackage{times}
\usepackage[french]{babel}
\usepackage{titlesec} % For chapter formatting
\usepackage{tikz}
\usepackage{fancyhdr}
\usepackage{graphicx} % includegraphics
\graphicspath{ {images/} }
\usepackage[
breaklinks=true,colorlinks=true,
linkcolor=black,urlcolor=black,citecolor=black,
bookmarks=true,bookmarksopenlevel=2]{hyperref}  % Links color in pdf
\usepackage{bookmark} % Creates bookmarks in pdf files
\usepackage[
  top=2.5cm,
  bottom=2.5cm,
  left=2.5cm,
  right=2.5cm,
  headsep=3pt,
  footskip=30pt,
  headheight=46pt,
  asymmetric
]{geometry}

% Header
\pagestyle{fancy}
\fancypagestyle{main}{%
  \lhead{\includegraphics[scale=0.5]{HEIG-VD}}
  \rhead{\sffamily Projet Réseau de Petri\\Bise Léonard\\Informatique Embarquée (ISEC)}
  \rfoot{Page \thepage}
  \cfoot{}
}
\fancypagestyle{plain}{%
  \lhead{\includegraphics[scale=0.5]{HEIG-VD}}
  \rhead{\sffamily Projet Réseau de Petri\\Bise Léonard\\Informatique Embarquée (ISEC)}
    \rfoot{Page \thepage}
      \cfoot{}
}

% Modifie la section chapitre
\titleformat{\chapter}
  {\sffamily\LARGE\bfseries}{\thechapter}{1em}{}
% spacing: how to read {12pt plus 4pt minus 2pt}
%           12pt is what we would like the spacing to be
%           plus 4pt means that TeX can stretch it by at most 4pt
%           minus 2pt means that TeX can shrink it by at most 2pt
%       This is one example of the concept of, 'glue', in TeX
\titlespacing*{\chapter}{0pt}{3.5ex plus 1ex minus .2ex}{2.3ex plus .2ex}
% Sets sections to sans serif
\setsecheadstyle{\LARGE\bfseries\sffamily}
\setsubsecheadstyle{\Large\bfseries\sffamily}
\setsubsubsecheadstyle{\large\bfseries\sffamily}
\setparaheadstyle{\large\bfseries\sffamily}
\setsubparaheadstyle{\large\bfseries\sffamily}

\begin{document}

%%%%%%%%%%%%%%%%%%%%% Page de Titre %%%%%%%%%%%%%%%%%%%%%%

\thispagestyle{empty}

\sffamily

% Logo HEIG-VD
\begin{tikzpicture}[remember picture,overlay]
\node[anchor=north west,yshift=-40pt,xshift=40pt]%
        at (current page.north west)
        {\includegraphics[scale=0.8]{HEIG-VD}};
\end{tikzpicture}

\large

~\vspace{\fill}

\begin{center}
{\HUGE 
Gestion d'un carrefour à l'aide d'un réseau de Petri
}
\vspace{1cm}
\hrule
\vspace{1cm}
{\LARGE
Rapport du projet
}
\end{center}

\vspace{3.5cm}

\begin{flushleft}
Auteur: Bise Léonard\\
Professeur: Zysman Eytan\\
Filière: Informatique Embarquée (ISEC)\\
Cours: Programmation Concurrente (PConc)\\
Date: 10 Mai 2018
\end{flushleft}

\vspace{\fill}

\cleardoublepage

%%%%%%%%%%%%%%%%%%%%%% Table des matières %%%%%%%%%%%%%%%%%%%%%%%

\tableofcontents

\clearpage

%%%%%%%%%%%%%%%%%%%%%% Liste des figures %%%%%%%%%%%%%%%%%%%%%%%

%\listoffigures  % Write out the List of Figures

%\clearpage

%%%%%%%%%%%%%%%%%%%%%% Liste des Tables %%%%%%%%%%%%%%%%%%%%%%%

%\listoftables  % Write out the List of Tables

%\clearpage

%%%%%%%%%%%%%%%%%%%%%% Introduction %%%%%%%%%%%%%%%%%%%%%%%

\chapter{Introduction}

Dans le cadre du cours Programmation Concurrente (PConc), il nous a été demandé de créer un programme qui simule le comportement d'un carrefour. Le carrefour est composé de deux flux, un Sud-Nord et l'autre Ouest-Est ou les voitures circulent dans un seul sens. A l'intersection des deux flux sont placé deux feux de signalisation, un par flux, ils permettent ou non au voiture de passer.\par
La gestion du système est fait grâce à un réseau de Petri qui réagit aux événement et qui produit des actions.

\chapter{Cahier des Charges}

Dans le cadre du projet il nous est demandé de créer un projet codé en java comprenant les éléments suivants:
\begin{itemize}
\item Créer un gestionnaire de réseau de pétri auquel il est possible de donner un fichier de configuration pour créer un paysage (création des places, transition, arcs de pre et post incidence). Une fois le paysage créé on peut lancer la simulation du réseau de pétri. Lorsque des jetons arrivent dans des places, des actions sont lancé au travers d'objet qui implémente une interface PetriPlaceAction. Le système peut déclencher des événements liés au réseau de petri qui permettent de franchir des transitions.
\item Utiliser le gestionnaire de réseau de pétri afin d'implémenter un réseau qui permet la gestion d'un carrefour avec deux flux perpendiculaire de voiture et qui est à sens unique. Des véhicules seront créé à des temps aléatoire par des threads dans les deux flux, ils avanceront en direction du carrefour, à leur arrivé devant le carrefour ils devront vérifier si le feu de signalisation est vert avant de pouvoir entrer dans le carrefour. Le RDP permettra, au travers d'actions liées à des places ainsi que d'événements, de changer la couleur des feux.
\end{itemize}

\chapter{Architecture}

L'image suivante montre l'architecture générale de l'application.

\includegraphics[scale=0.5]{images/architecture.png} 

\chapter{Le mécanisme du Réseau de Petri}

La classe PetriNetManager implémente le gestionnaire de réseau de Petri. D'autres classes sont utilisées pour la gestion des réseaux, elles sont décrites ci-dessous.\\
\begin{itemize}
\item PetriNetManagerTest: Contient quelques tests du bon fonctionnement du gestionnaire de réseau
\item PetriArc: Décrit les caractèristiques lié aux arcs de pré et post incidence comme le type (simple, test, inhibit) et leur poids
\item PetriPlace: Une place dans un réseau de Petri ainsi que sont action liée si elle en as une
\item PetriPlaceInterface: Une interface qui permet à la classe qui l'implémente d'être ensuite déclenchée par le gestionnaire de réseau lorsqu'un jeton entre dans la place qui y est liée.
\item PetriTransition: Contient les informations relative à une transition faisant partie du réseau de Petri\\
\end{itemize}
La classe PetriNetManager est la classe centrale du gestionnaire de réseau de Petri. Après l'instanciation de la classe, un réseau de Petri peut être initialisé en utilisant la méthode loadFromTextFile avec un fichier de configuration. Ceci créera toute les places, transitions et arcs requis pour le réseau décrit.\\\\
Une fois le réseau initialisé l'utilisateur à la possibilité de faire les actions suivantes:
\begin{itemize}
\item Lier une place du réseau avec une classe implémentant l'interface PetriPlaceInterface avec la méthode setPlaceAction. Lorsque le gestionnaire du réseau détecte l'entrée d'un jeton dans la place, il déclenchera automatiquement l'exécution de l'action
\item Lancer le gestionnaire de réseau de Petri avec la méthode start. Cette action lancera l'exécution du Thread de PetriNetManager qui après chaque tick effectuera la simulation du réseau
\item Une fois le gestionnaire lancé, l'utilisateur doit déclencher des événements au moyen de la méthode fireTransition qui permettre à l'utilisateur de déclencher l'événement lié à une certaine transition ce qui permettra le franchissement de la dite transition si elle est préalablement sensibilisée\\
\end{itemize}
Le coeur de la simulation du réseau de Petri est effectuée par la méthode step. Cette méthode simule un pas de  l'évolution du réseau de Petri, elle est cadencé par la durée d'un tick, c'est à dire un temps entre deux pas.\\
La liste suivante décrits les opérations effectuées par la méthode step, elle se découpe en quatre phases.\\
\begin{itemize}
\item Phase 0 : Les actions des places qui ont eu de nouveaux jetons sont exécutées
\item Phase 1 : Le gestionnaire détermine les transitions qui sont sensibilisées, c'est à dire que les places qui y sont liées contiennent le nombre de jeton requis, et les ajoute dans une liste de transition sensibilisée. La liste est ensuite mélangée pour que les transitions aient un ordre aléatoire
\item Phase 2 : Pour chaque tranisition qui sont dans la liste des transitions sensibilisées, le gestionnaire vérifie si l'événment associé a été déclenché. Si c'est le cas, les jetons des places qui sont liées à la transition sont consommé et elle est ajoutée à la liste des transitions franchies. La liste des transitions sensibilisées est à nouveau parcourue pour vérifier si certaines transitions ne le sont plus suite à la consommation de jeton
\item Phase 3: Les jetons nécessaire sont produit dans les places liées aux transitions qui ont été franchies
\end{itemize}

\chapter{Implémentation des acteurs}

Le projet de gestion de carrefour demande la création de différents acteurs qui sont listé ci-dessous.\\

\begin{itemize}
\item RoadCrossing: La grille sur laquelle les voitures se déplace. Cette classe est passive, elle ne fait que d'être modifié par d'autre classe, elle n'as pas son propre Thread.
\item RoadCrossingDetector: Les detecteurs utilisés pour déclencher des événements dans le réseau de Petri. Il y'en a deux par flux. Un placé juste avant le carrefour déclenchera l'événement lié à la détection de flux voulant entrer dans le carrefour, cela permet au réseau de Petri de faire passer le feu de l'autre flux au rouge et ainsi permettre au premier flux de pouvoir passer
\item RoadSignal: Le feu routier, c'est un élément passif qui ne fait que d'avoir un état, soit rouge ou alors vert. Il y a un feu par flux, les véhicules lorsqu'ils arrivent devant le carrefour doivent veiller à vérifier que le feu est vert avant de pouvoir y rentrer. L'état des feux est modifiés au travers de deux actions du réseau de Petri.
\item EventManager : C'est une classe qui permet de définir des événements réél et les associer a un événement du réseau de Petri en utilisant à l'interne des classe de type Event. C'est donc l'interface entre le monde réél et le réseau de Petri
\item RoadCrossingEventManager : Utilise la classe EventManager pour définir les événements qui sont propres au problème du carrefour (par exemple Génération d'un nouveau véhicule ou arrivée devant le carrefour)
\item SignalStateAction : Cette classe implémente l'interface PetriPlaceAction et défini donc l'action à effectuer lorsqu'un feu doit changer d'état. Elle directement appellé par le gestionnaire de réseau de Petri quand nécessaire et elle contient une réference au feu qu'elle doit gérer.
\item Vehicle : Définit le comportement d'un véhicule, après avoir été créé il commence à avancer régulièrement le long de son flux en utilisant la grille défini par la classe RoadCrossing. Juste avant de rentrer dans le carrefour le couleur du feu sera vérifier pour voir si il est possible d'avancer. Lorsqu'elle arrive à la fin de la route, le véhicule a terminé sa vie et est détruit.
\item VehicleCreator : Ce Thread est responsable de la création de véhicule dans un certain flux de manière régulière
\end{itemize}

\chapter{Les échanges entre acteurs}



%%%%%%%%%%%%%%%%%%%%%% Bibliographie %%%%%%%%%%%%%%%%%%%%%%%

%\bibliographystyle{unsrt}
%\bibliography{bibliographie}  % The references (bibliography) information are stored in the file named "bibliographie.bib"

\end{document}